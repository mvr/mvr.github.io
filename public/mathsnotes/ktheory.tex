\documentclass[10pt,a4paper]{article}

\usepackage[backend=bibtex]{biblatex}
\addbibresource{ktheory.bib}
\nocite{*}

\usepackage{amsmath}
\usepackage{amsfonts}
\usepackage{amssymb}
\usepackage{amsthm}
\usepackage{enumerate}
\usepackage[all, cmtip]{xy}

\DeclareMathOperator{\im}{im}
\DeclareMathOperator{\Hom}{Hom}

\theoremstyle{definition}
\newtheorem{theorem}{Theorem}[section]
\newtheorem{prop}[theorem]{Proposition}
\newtheorem{lemma}[theorem]{Lemma}
\newtheorem{example}[theorem]{Example}
\newtheorem{definition}[theorem]{Definition}

\author{Mitchell Riley}
\title{Understanding $K_0$}

\begin{document}
\maketitle

The study of K-theory began in 1957 in a paper of Grothendieck \cite{grothendieck1957}. This paper detailed a reformulation of the Riemann-Roch theorem, an important result in complex analysis and algebraic geometry. To do so, Grothendieck introduced $K(\mathcal{A})$, a group associated with a subcategory $\mathcal{A}$ of an abelian category. A few years later, Atiyah and Hirzebruch used a similar construction to define the K-theory $K_0(X)$ of vector bundles over a topological space $X$. Around the same time, algebraic K-theory was being developed to study the behaviour of projective modules over a ring $R$, leading to $K_0(R)$.

These examples are all instances of the same construction, first considered by Grothendieck. In this article we will consider the K-theory of a simpler structure -- symmetric monoidal categories -- and show that $K_0$ of a topological space and $K_0$ of a ring both fit this description.

\section{Group Completion of a Monoid}

The natural numbers $\mathbb{N}$ form an abelian monoid under addition. To construct the integers, we typically consider pairs of natural numbers $(m, n)$ under the equivalence $(m, n) \equiv (m+p, n+p)$. In a precise sense, the abelian group $\mathbb{Z}$ that we produce is the simplest extension of $\mathbb{N}$ to an abelian group. We wish to apply a similar construction to any abelian monoid, which leads us to the notion of group completion.

\begin{definition}
Given an abelian monoid $M$, the \emph{group completion} of $M$, denoted $M^{-1} M$, is an abelian group and monoid map $[ \cdot ] : M \to M^{-1}M$ satisfying the following universal property:

For every abelian group $A$ and monoid map $\alpha : M \to A$, there is a unique abelian group homomorphism $\tilde{\alpha} : M^{-1}M \to A$ such that $\tilde{\alpha}([m]) = \alpha(m)$.
\end{definition}

Every abelian monoid has a group completion, which we can construct as follows. Let $F(M)$ be the free abelian group on symbols the elements of $M$, and $R(M)$ the subgroup generated by $[m+n] - [m] - [n]$. The group completion $M^{-1} M$ is then the quotient of $F(M)$ by $R(M)$.

\begin{prop}
This construction satisfies the universal property.
\end{prop}
\begin{proof}
Let $\alpha$ be any monoid map $\alpha : M \to A$. Consider the abelian group homomorphism $g : F(M) \to A$ defined by extending $\alpha$ on the basis elements of $F(M)$. Because $\alpha$ is a monoid map, for any $r\in R(M)$, $g(r) = 0$, which implies $g(R(M)) = 0$, i.e., $R(M) \subseteq \ker{g}$.

Now consider the canonical homomorphism $\pi : F(M) \to F(M) / R(M)$. $\ker \pi = R(M) \subseteq \ker{g}$, so we have a unique map $\tilde{\alpha} : F(M) / R(M) \to A$ with $g = \tilde{\alpha} \circ \pi$.

Finally we can define $[ \cdot ] : M \to F(M) / R(M)$ as $[m] = (\pi \circ \iota)(m)$ where $\iota : M \to F(M)$ is the inclusion of $M$ as the basis elements of $F(M)$.

This satisfies the property $\tilde{\alpha}([m]) = (\tilde{\alpha} \circ \pi \circ \iota)(m) = (g \circ \iota)(m) = \alpha(m)$. $\alpha$ was arbitrary, so the universal property is satisfied and $F(M) / R(M) = M^{-1} M$ is the group completion of $M$.
\end{proof}

\begin{prop}
Let $M$ be an abelian monoid. Then:
\begin{enumerate}[a)]
\item Every element of $M^{-1} M$ is of the form $[m] - [n]$ for some $m, n \in M$
\item $[m] = [n]$ in $M^{-1} M$ iff $m+p = n+p$ in $M$ for some $p \in M$
\end{enumerate}
\end{prop}
\begin{proof}
Every element in the free abelian group $F(M)$ is a difference of sums of generators. We have that $([m_1] + [m_2] + \dots) \equiv [m_1 + m_2 + \dots] \text{ modulo } R(M)$. So every element in $M^{-1} M$ is a difference of generators. This establishes a).

For b), Suppose that $[m] - [n] = 0$. Then, in the group $F(M)$ we must have:
\begin{align*}
[m] - [n] = \sum ([a_i + b_i] - [a_i] - [b_i]) - \sum ([c_j + d_j] - [c_j] - [d_j])
\end{align*}
Rearranging:
\begin{align*}
[m] + \sum ([a_i] + [b_i]) + \sum ([c_j + d_j]) = [n] + \sum ([a_i + b_i]) + \sum ([c_j] + [d_j])
\end{align*}

Because $F(M)$ is free, these can only be equal if the same generators appear the same number of times on both the left and right. Therefore, in $M$,
\begin{align*}
m + \sum (a_i + b_i) + \sum (c_j + d_j) &= n + \sum (a_i + b_i) + \sum (c_j + d_j) \\
m + p &= n + p
\end{align*}
\end{proof}

Therefore, another way of thinking about the group completion is as the set quotient of $M \times M$ by the equivalence relation $(m, n) \sim (m + p, n + p)$.

Group completion is a functor from abelian monoids to abelian groups. Given a monoid map $M \to N$, we can consider monoid map $M \to N \to N^{-1} N$ given by composing with $[\cdot]$. By the universal property, we have an abelian group homomorphism $M^{-1}M\to N^{-1}N$.

\section{Symmetric Monoidal Categories}
\begin{definition}
A \emph{symmetric monoidal category} is a category $\mathcal{C}$ equipped with a functor $\boxdot : \mathcal{C} \times \mathcal{C} \to \mathcal{C}$, and a distinguished object $e$, where we have the natural isomorphisms:
\begin{align*}
  e \boxdot s \cong s, \quad
  s \boxdot e \cong s, \quad
  s \boxdot (t \boxdot u) \cong (s \boxdot t) \boxdot u, \quad
  s \boxdot t \cong t \boxdot s
\end{align*}
\end{definition}

We also require that these isomorphisms be `coherent', meaning that any two isomorphisms built out of the basic ones given above are the same whenever they have the same source and target. For example, we can build an isomorphism between $((s \boxdot t) \boxdot u) \boxdot v$ and $s \boxdot (t \boxdot (u \boxdot v))$ via the following two different paths, and we require that these isomorphisms are the same:
\begin{displaymath}
    \xymatrix{
        & (s \boxdot t) \boxdot (u \boxdot v) \ar[dr] & \\
		((s \boxdot t) \boxdot u) \boxdot v \ar[ru] \ar[d] & & s \boxdot (t \boxdot (u \boxdot v)) \\
		(s \boxdot (t \boxdot u)) \boxdot v \ar[rr] & & s \boxdot ((t \boxdot u) \boxdot v) \ar[u]
    }
\end{displaymath}

\begin{example}
  Any category with finite products or coproducts is symmetric monoidal under that operation, this includes the categories of sets ($s \boxdot t = s \sqcup t$ or $s \boxdot t = s \times t$), groups ($s \boxdot t = s \times t$) and vector spaces ($s \boxdot t = s \oplus t$).
\end{example}

\begin{definition}
Let $\mathcal{C}$ be a symmetric monoidal category under $\boxdot$, with the added condition that the isomorphism classes of $\mathcal{C}$ form a set, which we name $S^{\text{iso}}$. This set forms an abelian monoid under the operation $\boxdot$, with identity $e$. $K_0^{\boxdot}(S)$ is defined to be the group completion of this monoid.
\end{definition}

\begin{example}
Consider the category $\mathbf{Set}_\text{fin}$ of finite sets. This category is symmetric monoidal under the operation of disjoint union $\sqcup$. There is one isomorphism class for each natural number; the class contains all sets of that size. The size of the disjoint union of two finite sets is the sum of the sizes of the two sets. Therefore $K_0^{\sqcup}(\mathbf{Set}_\text{fin}) = \mathbb{N}^{-1}\mathbb{N} = \mathbb{Z}$.

We could also consider $\mathbf{Set}_\text{fin}$ with the Cartesian product. Because $A \times \emptyset = \emptyset$ we have that $A \times \emptyset = B \times \emptyset$ for all finite sets $A$ and $B$, so $K_0^{\times}(\mathbf{Set}_\text{fin}) = 0$.
\end{example}

\section{$K_0$ of a Topological Space}

To define $K_0$ of a topological space, we need some corresponding symmetric monoidal category. The appropriate category here is the category of vector bundles over that space.

\begin{definition}
A \emph{real vector bundle} over a space $X$ is a space $E$ together with a continuous map $\pi : E \to X$ such that for every $x \in X$, $\pi^{-1}(\{x\})$ has the structure of a real vector space. These vector spaces $\pi^{-1}(\{x\})$ are known as fibers. We also require that \emph{locally} $E$ looks like the direct product $X \times \mathbb{R}^k$.

More formally, for every point in $X$ there is a neighbourhood $U \subset X$, and a homeomorphism $\varphi : U \times \mathbb{R}^k \to \pi^{-1}(U)$ such that for all $x \in U$,
\begin{enumerate}
\item $(\pi \circ \varphi)(x, v) = x$
\item The map $v \mapsto \varphi(x, v)$ is an isomorphism of vector spaces.
\end{enumerate}

A \emph{complex vector bundle} has the same definition, using complex vector spaces.
\end{definition}

\begin{example}
On any topological space $X$ we can construct the \emph{trivial bundle} by taking $\epsilon^n = X \times \mathbb{R}^n$ or $\epsilon^n = X \times \mathbb{C}^n$ for any $n$.
\end{example}

\begin{example}
Given a differentiable manifold $M$, the disjoint union of the tangent spaces at each point forms a real vector bundle $TM$, called the tangent bundle of M.
\end{example}

We also need a notion of maps between vector bundles. A homomorphism from the vector bundle $\pi_1 : E_1 \to X$ to $\pi_2 : E_2 \to X$ is a continuous map $f : E_1 \to E_2$ such that:
\begin{enumerate}
\item $\pi_1 = \pi_2 \circ f$ (The fiber over a point maps into the fiber over the same point)
\item For every $x \in X$, the induced map $\pi_1^{-1}(\{x\}) \to \pi_2^{-1}(\{x\})$ is a linear map of vector spaces
\end{enumerate}

If this map has an inverse $E_2 \to E_1$ that is also a bundle homomorphism, we call this an isomorphism of vector bundle.

\begin{example}
Over $\mathbb{S}^1$, for example, we have the trivial bundle $\mathbb{S}^1 \times \mathbb{R}$. This is not the only 1-dimensional real bundle over $\mathbb{S}^1$, there is also the M\"{o}bius bundle defined as follows. Consider $\mathbb{S}^1$ as the unit circle in $\mathbb{C}$, then:
\begin{align*}
\mathcal{M} = \left\{ (e^{i \theta}, v) \in S \times \mathbb{R}^2 \: \middle| \: v = (\lambda \cos{\frac{\theta}{2}}, \lambda \sin{\frac{\theta}{2}}) \right\}
\end{align*}
Intuitively, the 1-dimensional subspace makes a half-turn as it travels around the circle. The M\"{o}bius bundle and trivial bundle are not isomorphic.
\end{example}

\begin{example}
The tangent bundle of a manifold can be isomorphic to the trivial bundle but this is not always the case. The tangent bundle of $\mathbb{S}^1$ is trivial but the tangent bundle of $\mathbb{S}^2$ is not. This last fact is related to the hairy ball theorem.
\end{example}

We can now consider the categories $\mathbf{VB}_\mathbb{R}(X)$ or $\mathbf{VB}_\mathbb{C}(X)$ of real or complex vector bundles over a fixed base space $X$, where the morphisms are given by vector bundle homomorphisms. The remaining piece is some bifunctor $\mathbf{VB}(X) \times \mathbf{VB}(X) \to \mathbf{VB}(X)$, and this is given by the Whitney sum.

\begin{definition}
The \emph{Whitney sum} of two vector bundles $\pi_1 : E_1 \to X$ and $\pi_2 : E_2 \to X$ is given by the pointwise direct sum of the fibers:
\begin{align*}
E_1 \oplus E_2 = \bigcup_{x \in X} (E_1)_x \oplus (E_2)_x
\end{align*}
This is given the subspace topology as a subspace of $E_1 \times E_2$. $E_1$ and $E_2$ are locally trivial, so $E_1 \oplus E_2$ is also.
\end{definition}

\begin{example}
The sum of the M\"{o}bius bundle with itself, $\mathcal{M} \oplus \mathcal{M}$, is isomorphic to the trivial bundle $\epsilon^2$.
\end{example}

The real K-theory of $X$, written $KO(X)$, is $K_0$ of the category of finite dimensional real vector bundles; $KO(X) = K_0^\oplus(\mathbf{VB}_\mathbb{R}(X))$. The complex K-theory of $X$ is the same for complex vector bundles; $K(X) = K_0^\oplus(\mathbf{VB}_\mathbb{C}(X))$.

\begin{example}
Let $*$ denote a one point space. The only vector bundles over this space are the trivial vector bundles for each dimension. Therefore:
\begin{align*}
KO(*) = K(*) = \mathbb{N}^{-1}\mathbb{N} = \mathbb{Z}
\end{align*}
\end{example}

\begin{example}
Real vector bundles on $\mathbb{S}^1$ are all either of the form $\epsilon^n$ or $\epsilon^n + \mathcal{M}$. When two are added, if they both contain a M\"{o}bius bundle these components will combine to form a 2-dimensional trivial bundle.

The group completion of this monoid structure is $KO(\mathbb{S}^1) = \mathbb{Z} \oplus \mathbb{Z}_2$
\end{example}

\begin{example}
The famous Bott periodicity theorem states that the K-theory of the spheres is periodic in the dimension, with period 2 for complex vector bundles and period 8 for real vector bundles:
\begin{align*}
K(\mathbb{S}^{2n}) &= \mathbb{Z} \oplus \mathbb{Z} \\
K(\mathbb{S}^{2n+1}) &= \mathbb{Z}
\end{align*}
\end{example}

\section{$K_0$ of a Ring}

Given a ring $R$, the relevant category is the category of finitely generated projective $R$-modules. We will need some preliminary definitions:

\begin{definition}
An $R$-module $M$ is \emph{free} if there is a subset $\{e_i\}$ such that every element of $M$ can be expressed uniquely as a finite sum $\sum r_i e_i$ with $r_i \in R$.
\end{definition}

\begin{example}
If $R$ is a field, from linear algebra we know that $R$-module has a basis, so every $R$-module is free.
\end{example}

\begin{example}
Not every $\mathbb{Z}$-module is free. Consider $\mathbb{Z}_2$. The only possible candidate for a basis is $\{1\}$, but then $1 \cdot 1 = 3 \cdot 1 = 1$, so we cannot express $1$ uniquely as a sum as required above.

The only free $\mathbb{Z}$-modules are the free abelian groups.
\end{example}

\begin{definition}
A \emph{projective $R$-module} is a module $P$ such that there exists another module $Q$ where $P \oplus Q$ is free.
\end{definition}

\begin{example}
Any free module is automatically projective. As an example of a projective module that is \emph{not} free, consider $\mathbb{Z}_2$ as an $\mathbb{Z}_6$-module. $\mathbb{Z}_2$ cannot be free, as a nontrivial free $\mathbb{Z}_6$-module necessarily has at least 6 elements.

However, $\mathbb{Z}_2 \oplus \mathbb{Z}_3 \cong \mathbb{Z}_6$ which is free, so $\mathbb{Z}_2$ is a projective $\mathbb{Z}_6$-module.
\end{example}

\begin{example}
Let $R = GL_2(F)$ with $F$ a field. Then the space of column vectors
\begin{align*}
P = \left\{ \begin{pmatrix} a \\ b \end{pmatrix} \: \middle| \:a, b \in F \right\}
\end{align*}
is a projective $R$-module, as $R = P \oplus P$. A similar idea works for any matrix ring $R = GL_n(F)$.
\end{example}

Finitely generated projective modules over $R$ form a category $\mathbf{P}(R)$, where the morphisms are module homomorphisms. Our binary operation is ordinary module direct sum $\oplus$. The K-theory of a ring is $K_0(R) = K_0^\oplus(\mathbf{P}(R))$.

\begin{example}
If $R$ is a field, the only isomorphism classes of finitely generated modules are the vector spaces of each dimension. Therefore, the monoid $\mathbf{P}(F)$ is isomorphic to $\mathbb{N}$ and $K_0(F) \cong \mathbb{Z}$.

Using the structure theorem for finitely generated modules over a PID, a similar argument shows that for any PID $R$, we also have that $K_0(R) \cong \mathbb{Z}$. In particular, $K_0(\mathbb{Z}) \cong \mathbb{Z}$.
\end{example}

In general, the K-theory of a ring can be very hard to compute!

\section{$K_0$ of an Abelian or Exact Category}

There is another setting in which we can define the Grothendieck group of a category, and that is when our category is an \emph{exact category}. First let us define $K_0$ for a slightly less general class of categories, abelian categories.

\begin{definition}
An \emph{additive category} is a category where
\begin{itemize}
\itemsep0em
\item The category contains a zero object $0$
\item All binary products and coproducts exist
\item Every set $\Hom(A, B)$ has the structure of an abelian group, and morphism composition is bilinear
\end{itemize}
\end{definition}

In an additive category the product and coproduct coincide; we call the operation the direct sum, written $A \oplus B$.

\begin{definition}
An \emph{abelian category} is an additive category satisfying the further conditions
\begin{itemize}
\itemsep0em
\item Every morphism $f$ has both a kernel and a cokernel
\item Every monomorphism is the kernel of some morphism, and every epimorphism is the cokernel of some morphism.
\end{itemize}
\end{definition}

\begin{example}
The prototypical example of an abelian category is $\mathbf{Ab}$, the category of abelian groups. If $R$ is a ring then the category of all $R$-modules, $R$-$\mathbf{Mod}$, is also an abelian category.
\end{example}

In fact, the Freyd-Mitchell embedding theorem states that every small abelian category $\mathcal{A}$ is equivalent to a full subcategory of $R$-$\mathbf{Mod}$ for some ring $R$.

\begin{example}
The category $\mathbf{VB}(X)$ is not an abelian category. The kernel of a map of vector bundles is not a vector bundle in any natural way.
\end{example}

Abelian categories are the natural setting in which to study exact sequences.

\begin{definition}
A sequence $A \xrightarrow{f} B \xrightarrow{g} C$ is \emph{exact} if $\im(f) = \ker(g)$. A longer sequence is exact if it is exact at every place.

A \emph{short exact sequence} is an exact sequence of the form
\begin{align*}
0 \to A \to B \to C \to 0
\end{align*}
\end{definition}

\begin{example}
The following is an exact sequence in $\mathbf{Ab}$.
\begin{align*}
0 \to \mathbb{Z} \xrightarrow{2\times} \mathbb{Z} \to \mathbb{Z}_2 \to 0
\end{align*}
\end{example}

Using this notion of exact sequence we can give a definition of the Grothendieck group of $\mathcal{A}$.

\begin{definition}
Let $\mathcal{A}$ be an abelian category. The Grothendieck group $K_0(\mathcal{A})$ is the abeilan group with generators $[A]$ for each object in $\mathcal{A}$, and relations $[B] = [A] + [C]$ for each short exact sequence $0 \to A \to B \to C \to 0$.
\end{definition}

Some simple consequences of this definition:
\begin{itemize}
\itemsep0em
\item $[0] = 0$, by considering the sequence $0 \to A \to A \to 0 \to 0$
\item If $A \cong B$, then $[A] = [B]$, by considering $0 \to A \to B \to 0 \to 0$
\item $[A \oplus B] = [A] + [B]$, by considering $0 \to A \to A \oplus B \to B \to 0$
\end{itemize}

The last point implies that $K_0(\mathcal{A})$ defined in this way is a quotient of $K_0^\oplus(\mathcal{A})$ considering $\mathcal{A}$ as a symmetric monoidal category with operation $\oplus$. The two are not necessarily isomorphic, as $\mathcal{A}$ may have other short exact sequences that do not split as $0 \to A \to A \oplus B \to B \to 0$.

\begin{example}
Consider the category $\mathbf{Ab}_p$ of finite abelian $p$-groups. $K_0(\mathbf{Ab}_p) \cong \mathbb{Z}$ with generator $[\mathbb{Z}_p]$. To show this, first let $l(M)$ denote the length of a composition series for a finite $p$-group $M$. $l$ induces a homomorphism $K_0(\mathbf{Ab}_p) \to \mathbb{Z}$, with $l([\mathbb{Z}_p]) = 1$.

Secondly, $[\mathbb{Z}_p]$ generates $K_0(\mathbf{Ab}_p)$. This follows by induction on $l(M)$, as for any $L \subset M$ we have $[M] = [L] + [M / L]$. So indeed $K_0(\mathbf{Ab}_p) \cong \langle [\mathbb{Z}_p] \rangle$.
\end{example}

\begin{example}
The direct sum of two abelian categories is also abelian, and $K_0(\mathcal{A}_1 \oplus \mathcal{A}_2) \cong K_0(\mathcal{A}_1) \oplus K_0(\mathcal{A}_2)$.

The category $\mathbf{Ab}_\text{fin}$ of all finite abelian groups is the direct sum of the categories $\mathbf{Ab}_p$. Therefore $\mathbf{Ab}_\text{fin} = \bigoplus \mathbf{Ab}_p$ is the free abelian group on the set $\{[\mathbb{Z}_p] : p \text{ a prime} \}$.
\end{example}

We do not need the full power of an abelian category to define $K_0(\mathcal{C})$. If $\mathcal{C}$ is an additive subcategory of an abelian category, the notion of exact sequence can still apply.

\begin{definition}
An \emph{exact category} is a pair $(\mathcal{C}, \mathcal{E})$ where $\mathcal{C}$ is an additive category and $\mathcal{E}$ is a family of exact sequences in $\mathcal{C}$ satisfying the following condition: there exists an embedding of $\mathcal{C}$ as a full subcategory of an abelian category $\mathcal{A}$ so that
\begin{itemize}
\itemsep0em
\item $\mathcal{E}$ is the class of all sequences in $\mathcal{C}$ that are exact in $\mathcal{A}$
\item If there is an exact sequence in $\mathcal{A}$, $0 \to A \to B \to C \to 0$ with $A, C \in \mathcal{C}$, then $B \in \mathcal{C}$
\end{itemize}
\end{definition}

The definition of $K_0(\mathcal{C})$ is the same as for an abelian category.

\begin{definition}
Let $\mathcal{C}$ be an exact category. The Grothendieck group $K_0(\mathcal{C})$ is the abeilan group with generators $[C]$ for each object in $\mathcal{C}$, and relations $[B] = [A] + [C]$ for each short exact sequence $0 \to A \to B \to C \to 0$ in $\mathcal{E}$.
\end{definition}

\begin{example}
The category $\mathbf{P}(R)$ is exact, given its embedding into $R$-$\mathbf{Mod}$. Every exact sequence of projective modules splits, so $K_0(\mathbf{P}(R)) = K_0(R)$ as defined in the previous section.
\end{example}

\begin{example}
The category $\mathbf{VB}(X)$ can be embedded in the category of sheaves of $O_X$-modules. Again, every exact sequence of vector bundles splits so $K_0(\mathbf{VB}(X)) = K(X)$.
\end{example}

\section{$K_0$ of a Waldhausen Category}

We can weaken our conditions on the category further and still have a notion of $K_0(\mathcal{C})$.

\begin{definition}
A \emph{Waldhausen category} $(\mathcal{C}, co, we)$ is a category $\mathcal{C}$ with a zero object together with two distinguished classes of morphisms; $co$, the cofibrations $A \rightarrowtail B$, and $we$, the weak equivalences $C \xrightarrow{\sim} D$. These distinguished morphisms satisfy the following conditions.
\begin{itemize}
\itemsep0em
\item All isomorphisms are cofibrations
\item For any object $A$ the unique morphism $0 \to A$ is a cofibration
\item If $A \rightarrowtail B$ is a cofibration and $A \to C$ any morphism then the pushout $C \rightarrowtail B \cup_A C$ is a cofibration
\\
\item All isomorphisms are weak equivalences
\item Weak equivalences are closed under composition
\item Given any commutative diagram of the form
\begin{displaymath}
    \xymatrix{
    C \ar[d]^\sim & \ar[l] A \ar[d]^\sim \ar@{>->}[r] & B \ar[d]^\sim \\
    C' & \ar[l] A' \ar@{>->}[r] & B' \\
    }
\end{displaymath}
the induced map $B \cup_A C \xrightarrow{\sim} B' \cup_{A'} C'$ is a weak equivalence.
\end{itemize}
\end{definition}

Informally, cofibrations are analogous to monomorphisms. A category with cofibrations is a category with a good notion of a monomorphism that is stable under pushouts.

\begin{example}
Waldhausen categories generalise the exact categories of last section. The cofibrations in an exact category $\mathcal{C}$ are the maps $i$ such that $i$ occurs in a short exact sequence $0 \to A \xrightarrow{i} B \to C \to 0$. The weak equivalences are the isomorphisms of $\mathcal{C}$.
\end{example}

\begin{example}
A Waldhausen category need not be additive. Let $\mathcal{R}_f$ be the category of based CW complexes with finitely many cells, where the morphisms are cellular maps. The cofibrations are the cellular inclusions, and the weak equivalences are the weak homotopy equivalences.
\end{example}

The notion of a quotient object is valid in a Waldhausen category. Let $f : A \rightarrowtail B$ be a cofibration in $\mathcal{C}$. By the 3rd condition, the pushout of $f$ and $A \to 0$ exists, which we will denote by $B/A = 0 \cup_A B$. We therefore have a diagram
\begin{displaymath}
    \xymatrix{
    A \ar@{>->}[r] \ar[d] & B \ar[d] \\
    0 \ar[r] & B/A
    }
\end{displaymath}
The sequence $A \rightarrowtail B \to B/A$ is called a cofibration sequence.

\begin{definition}
Let $\mathcal{C}$ be a Waldhausen category. The Grothendieck group $K_0(\mathcal{C})$ is the abeilan group with generators $[C]$ for each object of $\mathcal{C}$, and relations 
\begin{itemize}
\itemsep0em
\item $[C] = [C']$ for each weak equivalence $C \xrightarrow{\sim} C'$, 
\item $[B] = [A] + [B/A]$ for every cofibration sequence $A \rightarrowtail B \to B/A$.
\end{itemize}
\end{definition}

\begin{example}
By construction, for an exact category $\mathcal{C}$, the Waldhausen $K_0(\mathcal{C})$ agrees with $K_0(\mathcal{C})$ from the previous section.
\end{example}

\begin{example}
Consider $\mathcal{R}_f$ as defined above. The disk $D^n$ is homotopy equivalent to a point, so $[D^n] = 0$. The inclusion of $S^{n-1}$ into $D^n$ has $D^n/S^{n-1} \cong S^n$, so $[S^{n-1}] + [S^n] = [D^n] = 0$. Therefore $[S^n] = (-1)^n [S^0]$.

If the CW complex $C$ is constructed from $B$ by attaching an $n$-cell, $C/B \cong S^n$ and $[C] = [B] + [S^n]$, and therefore $K_0(\mathcal{R}_f)$ is generated by $[S^0]$.

Finally, the reduced Euler characteristic $\chi(C)$ induces a surjection $K_0(\mathcal{R}_f)$ onto $\mathbb{Z}$, which must be an isomorphism.
\end{example}

\newpage
\printbibliography
\end{document}
