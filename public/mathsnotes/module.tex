\documentclass[10pt,a4paper]{article}
\usepackage[utf8]{inputenc}
\usepackage{amsmath}
\usepackage{amsfonts}
\usepackage{amssymb}
\usepackage{amsthm}

\newtheorem{definition}{Definition}
\newtheorem{theorem}{Theorem}
\newtheorem{proposition}{Proposition}
\newtheorem{corollary}{Corollary}
\newtheorem{lemma}{Lemma}

\relpenalty=9999
\binoppenalty=9999

\author{Mitchell Riley}
\title{The module perspective on representation theory}

\begin{document}
\maketitle

Modern representation theory looks at representations from a different perspective than the homomorphisms $\rho : G \to GL(V)$ we have been studying so far. Every representation turns out to be equivalent to a module over a particular algebra called the group algebra.

First let us review the structures we will be using.

\begin{definition}
A \emph{left $R$-module} is an abelian group $M$ with a left $R$ action such that for all $r, s \in R$ and $x, y \in M$
\begin{itemize}
\itemsep0em
\item $r(x+y) = rx + ry$
\item $(r+s)x = rx + sx$
\item $(rs)x = r(sx)$
\item $1x = x$
\end{itemize}
We say $M$ is \emph{finitely generated} if there is a finite set $\{x_i\}$ in $M$ such that every element of $M$ is an $R$-linear combination of these elements.
\end{definition}

Many familiar structures are modules over some ring. If we take $R$ to be a field, the above conditions are exactly the conditions of a vector space. In other words, a vector space over $\mathbb{F}$ is just an $\mathbb{F}$-module.

For any abeilan group $M$, $M$ is a $\mathbb{Z}$-module via the action $nx = \underbrace{x + \dots + x}_{n \text{ times}}$. This is why we sometimes hear abelian groups referred to as $\mathbb{Z}$-modules.

Finally, for any ring $R$, $R^n = R \oplus \dots \oplus R$ is a $R$-module, with $R$ acting component-wise: $rx = r(x_1, \dots, x_n) = (r x_1, \dots, r x_n)$. More generally we could consider $R^n$ as a module over the matrix ring $M_n(R)$, where the action is matrix multiplication on the column vector in $R^n$. When $R$ is a division algebra, any module over $M_n(R)$ is the direct sum of $m$ copies of $R^n$, a fact we will need later.

\begin{definition}
An \emph{algebra over a field $\mathbb{F}$} is a vector space over $\mathbb{F}$ with a bilinear product. In the case the product is associative, this algebra is also a ring.
\end{definition}

Every field is an algebra over itself, with the product given by normal multiplication in the ring. The complex numbers $\mathbb{C}$ are an algebra over $\mathbb{R}$, with basis $\{1, i\}$. The polynomial ring $\mathbb{F}[x]$ is an algebra over $\mathbb{F}$ with basis $\{x^0, x^1, \dots\}$ and product given by polynomial multiplication.

\begin{definition}
The \emph{group algebra} $\mathbb{F}[G]$ is the algebra with basis $\{g_i\}$ the elements of $G$, and product given by extending the group operation linearly. This product is clearly associative, so $\mathbb{F}[G]$ can also be considered as a ring.
\end{definition}

Every element of $\mathbb{F}[G]$ is of the form $\sum_{g \in G} a_g \cdot g$ for $a_g \in \mathbb{F}$, and multiplication is given by 
\begin{align*}
ab = (\sum_{g \in G} a_g \cdot g)(\sum_{h \in G} b_h \cdot h) = (\sum_{g \in G}\sum_{h \in G} a_g b_h \cdot gh)
\end{align*}

With these definitions out of the way we can show there is a correspondence between linear representations of $G$ and finitely generated $\mathbb{F}[G]$-modules. We will make use of the following fact.

\begin{lemma}
If $V$ is a finitely generated $\mathbb{F}[G]$-module, $V$ can be regarded as a finite dimensional vector space over $\mathbb{F}$ via $\lambda x = (\lambda \cdot 1)x$ where $1$ is the identity element of $G$.
\end{lemma}

\begin{proof}
If $V$ is generated as an $\mathbb{F}[G]$-module by $\{v_1, \dots, v_n\}$, then $V$ is generated as a $\mathbb{F}$-vector space by $\{g v_i \:|\: g \in G, 1 \leq i \leq t\}$. Because $G$ is finite, we have that this is finite dimensional.
\end{proof}

\begin{proposition}
Given a field $\mathbb{F}$ and a finite group $G$ there is a bijective correspondence between linear representations of $G$ and $\mathbb{F}[G]$-modules.
\end{proposition}

\begin{proof}
($\Rightarrow$) Given a representation $\rho : G \to GL(V)$, we can give $V$ an $\mathbb{F}[G]$-module structure through the action
\begin{align*}
	av = \left(\sum_{g \in G} a_g g\right)v = \sum_{g \in G} a_g (\rho_g v)
\end{align*}

($\Leftarrow$) Given a finitely generated $\mathbb{F}[G]$-module $V$, we know from the above lemma that $V$ is finite dimensional as a $\mathbb{F}$-vector space. The action is a map $\mathbb{F}[G] \times V \to V$. This restricts to a map $G \times V \to V$ that is also linear so equivalent to a representation $\rho : G \to GL(V)$.
\end{proof}

Considering representations as modules gives us access to some high-powered machinery from the theory of semisimple algebras. Firstly,

\begin{theorem}
If $\mathbb{F}$ has characteristic $0$, then $\mathbb{F}[G]$ is semisimple.
\end{theorem}

\begin{proof}
Analogous to Maschke's Theorem.
\end{proof}

A simple module is one with only the trivial module and itself as submodules. A simple $\mathbb{F}[G]$-module then corresponds to an irreducible representation. A semisimple module is one that decomposes as a direct sum of simple modules.

The theorem states that all $\mathbb{F}[G]$-modules decompose as direct sums of simple modules, or equivalently, all representations decompose into direct sums of irreducible representations. This is an alternative phrasing of Maschke's theorem.

$\mathbb{F}[G]$ being semisimple also gives us access to the following result.

\begin{theorem}[Wedderburn's theorem]
If $\mathbb{F}$ is algebraically closed, then any semisimple algebra is isomorphic to a direct sum of matrix algebras $M_i(\mathbb{F})$.
\end{theorem}

Because of the above condition, from here on we will take our field to be $\mathbb{C}$. The theorem states that $\mathbb{C}[G] \cong M_{n_1}(\mathbb{C}) \oplus \dots \oplus M_{n_r}(\mathbb{C})$ as $\mathbb{C}$-algebras. The theory of semisimple algebras now gives us alternative proofs of results we have already seen.

\begin{corollary}
There are exactly $r$ isomorphism classes of simple $\mathbb{C}[G]$-modules. These correspond to the irreducible representations, and the $n_i$ correspond to the degrees of these representations. Given representatives $S_1, \dots, S_r$ of these isomorphism classes, any $\mathbb{C}[G]$-module can be written uniquely in the form $V \cong a_1 S_1 \oplus \dots \oplus a_r S_r$ for some non-negative integers $a_i$.
\end{corollary}

\begin{proof}
As $\mathbb{C}$ is a division algebra, $\mathbb{C}^{n_i}$ is the only simple $M_{n_i}(\mathbb{C})$-module. Furthermore, because $M_{n_i}(\mathbb{C})$ is semisimple we know that $M_{n_i}(\mathbb{C})$-modules must all be of the form $a_i \mathbb{C}^{n_i}$ for some non-negative $a_i$. 

$\mathbb{C}[G]$ is the direct sum of such $M_{n_i}(\mathbb{C})$ so the result follows.
\end{proof}

\begin{corollary}
$\sum_i n_i^2 = |G|$
\end{corollary}
\begin{proof}
\begin{align*}
	|G| = \dim_{\mathbb{C}} \mathbb{C}[G] = \dim_{\mathbb{C}} \left(\bigoplus_i M_{n_i}(\mathbb{C})\right) = \sum_i \dim_{\mathbb{C}} M_{n_i}(\mathbb{C}) = \sum_i n_i^2
\end{align*}
\end{proof}

The final result I will revisit is the following:

\begin{proposition}
The irreducible characters of $G$, $\chi_i : G \to \mathbb{C}$, are linearly independent over $\mathbb{C}$.
\end{proposition}

\begin{proof}
Let $S_i$ be the simple $\mathbb{C}[G]$-modules and denote by $e_i$ the identity element of $M_{n_i}(\mathbb{C})$. $\chi_i(g)$ is the trace of the linear transformation on $S_i$ given by the action of $g$. We extend $\chi_i$ linearly to be a function $\mathbb{C}[G] \to \mathbb{C}$. The action of $e_i$ on $S_i$ is the identity, so we have that $\chi_i(e_i) = \dim_{\mathbb{C}} S_i = f_i$. For any $j \neq i$, the action of $e_i$ on $S_j$ is the zero map, so $\chi_j(e_i) = 0$.

Now let $\lambda_1, \dots, \lambda_r \in \mathbb{C}$ such that $\sum_j \lambda_j \chi_j = 0$. From the above we see that $0 = \sum_j \lambda_j \chi_j(e_i) = \lambda_i f_i$. Therefore, $\lambda_j = 0$ for all $j$, and the $\chi_i$ are all linearly independent.
\end{proof}

\end{document}